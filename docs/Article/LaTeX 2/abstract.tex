\begin{abstract}
{ Artificial Intelligence (AI) can be defined as a developed technology which has come to supplant cyclic tasks in different work environments. It allows the optimization of operational costs and productivity.  Thanks  to  it,  there  have  been  published  several  studies  in  different  areas:  as  an evaluation procedure to define the probability of guilty in legal processes; in clinical studies to develop and simulate new drug composition, and to predict electrocardiogram behaviour of heart problem  patients;  in  the  automotive  industry,  in  the  automation  of  production  systems  and tasks, due to the product variants and complexity, leading to simplify manufacturing processes;and in the design of photo-voltaic systems in modelling, sizing, control and diagnosis stages.
One  of  the  most  affected,  and  challenging,  industries  around  the  world  is  the  civil  industry, principally at the design stage. To optimize the process of design, structural and cost analysis, we created the iStructure project, which is a manufacturing tool that automate the design of structural members. It is a python-based framework library that, in its first version, can automate the design procedure of Mezzanine Floor Racking Systems. It allows the user to: specify the cross-section geometries of the structural members and predicts its geometric properties by the Finite Element Method – FEM; define the modular area distribution per floor; select the minimum cost cross-sections of the structural members which can resist the specified load conditions; create an automatic report, in LaTeX/PDF, which illustrates the design procedure (according to ANSI MH16.1, AISI S100 and ASCE 7-16 North American standards) and the costs of the structure; and elaborates automatic 3D CAD plans in DXF format. }
\end{abstract}

Keywords: Artificial Intelligence, Neural Networks, FEM, structural design, Mezzanine Floor Racking Systems, Open Source.