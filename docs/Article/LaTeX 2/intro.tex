\section*{INTRODUCTION}

Automation in engineering is widely applied to save both time and money. The construction industry is falling behind others in terms of making productivity gains\cite{Chen2018}. Various reasons have been invoked to explain negative productivity trends, such as shifts within construction, increases in land-use regulation and the use of questionable deflators\cite{ConsProd2016}. 

Unlike concrete buildings and bridges, metallic structures does not require special architectural design because its purpose is basically to increase the storage area of goods and raw materials of different industries in warehouses. Thanks to it, there is an innovation opportunity in the automation of the design stage of metallic structures. The present work focus on Mezzanine Floor Racking Systems and in the explanation of the logic behind a structural design framework, made by the authors, in Python - Jupyter, which starts by asking the user the area of the structure and load conditions. Then, it selects the structural members (beams, columns, and joists), evaluated by the FEM and selected by a defined design procedure, based on AISI S-100-16, ANSI MH16.1 and ASCE 7-16 North American standards and NSR-10 Colombian standard for seismic evaluation of structures installed in Colombia. Finally, the results are an economic report, an engineering design report and the CAD draws, 2D and 3D, of the structure.